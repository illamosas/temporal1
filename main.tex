\documentclass[letterpaper,12pt]{article}
\usepackage{tabularx} % extra features for tabular environment
\usepackage{amsmath}  % improve math presentation
\usepackage{graphicx} % takes care of graphic including machinery
\usepackage[margin=1in,letterpaper]{geometry} % decreases margins
\usepackage{cite} % takes care of citations
\usepackage[final]{hyperref} % adds hyper links inside the generated pdf file
\hypersetup{
	colorlinks=true,       % false: boxed links; true: colored links
	linkcolor=blue,        % color of internal links
	citecolor=blue,        % color of links to bibliography
	filecolor=magenta,     % color of file links
	urlcolor=blue         
}
\usepackage{blindtext}
%++++++++++++++++++++++++++++++++++++++++


\begin{document}

\date{\today}
\title{Phys3201 Template GT}
\author{A. Partner, B. Partner, and C. Partner}


\maketitle

\begin{abstract}
In this experiment we studied pendulum motion in a non-uniformly 
accelerating reference frame. \textbf{Special notes:} Prof Goldman 
specifically requested to not have ``100" "sig" figs, put font on 
figure axes in readable size, and don't put grids on plots! Y mi
entendimiento de latex es casi del 0\% despues de esta linea.

Seguir escribiendo. $$y_{i,t}= \alpha + \beta x_{i,t} + \varepsilon_{i,t}$$
$$\sum_i^j x_i$$
\end{abstract}


\section{Introduction}

Motivate why you chose the problem that you did. Why is it interesting? 

\subsection{subseccion de la clase}

\blindtext %delete this line


\section{Background}

Give a brief summary of the physical theory, include any equations necessary, and cite any references you want to include. Here is how you insert an equation. According to references ~\cite{melissinos, Cyr, Wiki} the dependence of interest is given
by


\begin{gather*}
   \mathcal{L} =  \frac{1}{2} m \ell^2 ( \dot{\theta}+\dot{\phi}_0)^2 - m g_e(t) \ell \cos(\theta)\\
   \\
   m\ell^2 (\ddot{\theta} + \ddot{\phi}_0) = mg_e\ell\sin(\theta)\\
\end{gather*}


\begin{equation}
   \dot{\phi}(t) = -\frac{g_e(t)}{\ell} \sin\left(\phi(t)-\phi_0(t)\right)
   \label{Eq:equation1} %the label lets you refer to the equation later
\end{equation}

\begin{equation}
   y_{i,t} = \alpha + \beta X_{i,t} + \varepsilon_{i,t}
   \label{Eq:irving} %the label lets you refer to the equation later
\end{equation}

Este es un texto de prueba, aqui voy a referenciar las ecuaciones. Por
ejemplo, en la ecuación \ref{Eq:equation1} podemos ver bla bla bla.
En otra ecuación, la numero \ref{Eq:irving} podemos ver lo contrario...


\begin{figure}[!h]
    \centering
    \includegraphics[width=.8\textwidth]{pendulum.jpg}
    \caption{Pendulum that starts at rest in an accelerating frame. If the acceleration is not constant then the apparent vertical, and thus $\phi_0$ will change with time}
    \label{fig:samplesetup}
\end{figure}

\blindtext %delete this line

\section{Methods}


Give a schematic of the experimental setup(s) used in the experiment (see
figure~\ref{fig:samplesetup}). Give the description of  abbreviations
either in the figure caption or in the text. Write a description of what is
going on. 

%\begin{figure}[ht] 
        % read manual to see what [ht] means and for other possible options
%        \centering \includegraphics[width=0.8\columnwidth]{sr_setup}
        % note that in above figure file name, "sr_setup",
        % the file extension is missing. LaTeX is smart enough to find
        % apropriate one (i.e. pdf, png, etc.)
        % You can add this extention yourself as it seen below
        % both notations are correct but above has more flexibility
        %\includegraphics[width=1.0\columnwidth]{sr_setup.pdf}
%        \caption{
%                \label{fig:samplesetup} % spaces are big no-no withing labels
                % things like fig: are optional in the label but it helps
                % to orient yourself when you have multiple figures,
                % equations and tables
%                Every figure MUST have a caption.
%        }
%\end{figure}

and eventually arrived to the
balanced photodiode as seen in the figure~\ref{fig:samplesetup}.


\section{Results}

In this section you will need to show your experimental results. Use tables and
graphs when it is possible. Table~\ref{tbl:bins} is an example.

\begin{table}[ht]
\begin{center}
\caption{Every table needs a caption.}
\label{tbl:bins} % spaces are big no-no withing labels
\begin{tabular}{|cc|} 
\hline
\multicolumn{1}{|c}{$x$ (m)} & \multicolumn{1}{c|}{$V$ (V)} \\
\hline
0.0044151 &   0.0030871 \\
0.0021633 &   0.0021343 \\
0.0003600 &   0.0018642 \\
0.0023831 &   0.0013287 \\
\hline
\end{tabular}
\end{center}
\end{table}

Analysis of equation~\ref{Eq:equation1} shows ...

\blindtext

\begin{figure}
    \centering
    \includegraphics[width=.7\textwidth]{tanh_vs_const_1.png}
    \caption{Hyperbolic tangent acceleration vs immediate constant acceleration. The slow approach to the same asymptotic value of 2 meters per second per second induces a lag in the oscillation and also diminishes the amplitude of oscillation.}
\end{figure}

For example, it is easy to conclude that the
experiment and theory match each other rather well if you look at
Fig.~\ref{fig:samplesetup} and Fig.~\ref{fig:exp_plots}.


\section{Conclusions}
Here you briefly summarize your findings. Did you learn any new physics? Was everything as expected?

\blindtext

\section{Future Work}
Since you had limited time to work on this project, what questions are left outstanding? What would be your next steps? 

\blindtext

%++++++++++++++++++++++++++++++++++++++++
% References section will be created automatically 
% with inclusion of "thebibliography" environment
% as it shown below. See text starting with line
% \begin{thebibliography}{99}
% Note: with this approach it is YOUR responsibility to put them in order
% of appearance.

\begin{thebibliography}{99}

\bibitem{melissinos}
A.~C. Melissinos and J. Napolitano, \textit{Experiments in Modern Physics},
(Academic Press, New York, 2003).

\bibitem{Cyr}
N.\ Cyr, M.\ T$\hat{e}$tu, and M.\ Breton,
% "All-optical microwave frequency standard: a proposal,"
IEEE Trans.\ Instrum.\ Meas.\ \textbf{42}, 640 (1993).

\bibitem{Wiki} \emph{Expected value},  available at
\texttt{http://en.wikipedia.org/wiki/Expected\_value}.

\end{thebibliography}


\end{document}
